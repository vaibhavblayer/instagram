\documentclass{article}
\usepackage{vaibhavblayer}
\instagramp

\header{200622}{MAG}{01}[E]
\footer{2}

\begin{document}

\pagecolor{white!85!orange}
\color{matte-black}

\title{Magnetic force on a moving charge}

\vspace*{\fill}
\begin{center}
\begin{tikzpicture}[>=latex, thick, line cap=round]
\fill (0, 0, 0) circle(2pt) node[below]{$q$};
\draw[->] (0,0,0)--(2,0,0) node[below]{$\vec{v}$};
\draw[->] (0,0,0)--(0,1.5,0) node[left]{$\vec{F}_m$};
\draw[->] (0,0,0)--(1,0,-2) node[above]{$\vec{B}$};
\draw (0.5,0,0) arc[start angle=0, end angle=24, radius=0.5];
\draw [line width=0 mm, opacity=0.35,pattern=grid, pattern color=matte-black] (0,0,0)--(2,0,0)--(3,0,-2)--(1,0,-2)--(0,0,0);
\end{tikzpicture}
\end{center}

\vspace*{\fill}
{\large
\[
\vec{F}_m = q \left( \vec{v} \times \vec{B} \right) 
\]
}

\vspace*{\fill}
\begin{equation*}
\eqnmarkbox{fm}{\vec{F}_m}
\eqnmarkbox{q}{q}
\eqnmarkbox{v}{\vec{v}}
\eqnmarkbox{b}{\vec{B}}
\end{equation*}
\annotate[yshift=-0.2 em]{below, left}{fm}{\texttt{magnetic force}}
\annotate[yshift=-1.2 em]{below, left}{q}{\texttt{charge}}
\annotate[yshift=-1.2 em]{below, right}{v}{\texttt{velocity}}
\annotate[yshift=-0.2 em]{below, right}{b}{\texttt{magnetic field}}
\end{document}